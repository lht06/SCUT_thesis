\chapter{摘\texorpdfstring{\quad}{}要}
本实训报告基于openGauss数据库系统的并行物化算子性能优化项目,重点分析算子在大规模数据处理场景下的性能表现与优化策略。

通过对物化算子的深入分析,识别出原始单线程实现在处理大数据集时存在CPU利用率低、内存访问效率差等关键瓶颈。针对这些问题,设计并实现了基于无锁环形缓冲区和动态负载均衡的并行优化方案。

实验采用多维度测试方法,包括基准性能测试、并发压力测试和长时间稳定性测试。使用火焰图工具深入分析系统性能瓶颈,对比基线版本和优化版本的执行热点分布。测试覆盖200万行、1000万行和2000万行三个数据规模,结果显示:在1000万行数据集上执行时间为2382.9ms;在2000万行数据集上执行时间为4756.1ms,相比原来的版本有可观的提升。并发测试表明系统在高负载下表现出良好的扩展性和稳定性。

\keywordsCN{数据库;物化算子;并行计算;性能优化;openGauss}

